% In this file you should put the actual content of the blueprint.
% It will be used both by the web and the print version.
% It should *not* include the \begin{document}
%
% If you want to split the blueprint content into several files then
% the current file can be a simple sequence of \input. Otherwise It
% can start with a \section or \chapter for instance.

\section{Introduction}
We follow the paper ``On ordered semigroups'' by
N. G. Alimov.

\section{Content}

\begin{definition}\label{def:OrderedSemigroup}\lean{OrderedSemigroup}\leanok
An \textbf{ordered semigroup} $S$ is a semigroup with a partial order such that
for all $a,b,c\in S$ such that $a \le b$
\begin{enumerate}
    \item $a * c \le b * c$ and
    \item $c * a \le c * b$
\end{enumerate}
\end{definition}

\begin{definition}\label{def:OrderedCancelSemigroup}\lean{OrderedCancelSemigroup}\leanok
\uses{def:OrderedSemigroup}
An \textbf{ordered cancel semigroup} $S$ is an ordered semigroup such that for all $a,b,c\in S$,
if $a * b \le a * c$ then $b \le c$ and if $b * a \le c * a$ then $b \le c$.
\end{definition}

\begin{definition}\label{def:LinearOrderedSemigroup}\lean{LinearOrderedSemigroup}\leanok
\uses{def:OrderedSemigroup}
A \textbf{linear ordered semigroup} is an ordered semigroup where its partial order is a linear order.
\end{definition}

\begin{definition}\label{def:LinearOrderedCancelSemigroup}\lean{LinearOrderedCancelSemigroup}\leanok
\uses{def:OrderedCancelSemigroup}
A \textbf{linear ordered cancel semigroup} is an ordered cancel semigroup where its partial order is a linear order.
\end{definition}

\begin{definition}\label{def:positive}\lean{is_positive}\leanok
\uses{def:OrderedSemigroup}
An element $a$ of an ordered semigroup $S$ is \textbf{positive} if for all $x\in S$, $a*x > x$.
\end{definition}

\begin{definition}\label{def:negative}\lean{is_negative}\leanok
\uses{def:OrderedSemigroup}
An element $a$ of an ordered semigroup $S$
is \textbf{negative} if for all $x\in S$, $a*x < x$.
\end{definition}

\begin{definition}\label{def:one}\lean{is_one}\leanok
\uses{def:OrderedSemigroup}
An element $a$ of an ordered semigroup $S$
is \textbf{one} if for all $x\in S$, $a*x = x$.
\end{definition}

\begin{theorem}\label{thm:pos_neg_or_one}\lean{pos_neg_or_one}\leanok
\uses{def:positive, def:negative, def:one, def:LinearOrderedCancelSemigroup}
Each element $a$ of a linear ordered cancel semigroup $S$ is either positive, negative, or one.
\end{theorem}
\begin{proof}
Let $a\in S$ and $b\in S$. Since the $S$ is a linear order we have one of the following cases.

The first case is that $b * a > b$. Then for all $x\in S$ we have that $b * a * x > b * x$
and so $a * x > x$. Therefore, $a$ is positive.

The second case is that $b * a < b$. Then for all $x\in S$ we have that $b * a * x < b * x$
and so $a * x < x$. Therefore, $a$ is negative.

The last case is that $b * a = b$. Then for all $x\in S$ we have that $b * a * x = b * x$
and so $a * x = x$. Therefore, $a$ is zero.
\end{proof}

\begin{corollary}\label{lem:right_forall}\lean{pos_right_pos_forall, neg_right_neg_forall, one_right_one_forall}\leanok
\uses{def:positive, def:negative, def:one}
Let $a$ be an element of a linear ordered cancel semigroup $S$. 

If there exists a $b\in S$ such that $b * a > b$, $a$ is positive.

If there exists a $b\in S$ such that $b * a < b$, $a$ is negative.

If there exists a $b\in S$ such that $b * a = b$, $a$ is one.
\end{corollary}

\begin{theorem}\label{thm:neg_lt_pos}\lean{neg_lt_pos}\leanok
\uses{def:positive, def:negative}
Let $a$ and $b$ be elements of a linear ordered cancel semigroup $S$.

If $a$ is negative and $b$ is positive, then $a < b$.
\end{theorem}
\begin{proof}
Let $a$ be negative and $b$ be positive. Then for all $x\in S$ we have that $a * x < x$.
Likewise, for all $x\in S$ we have that $b * x > x$.
Therefore, $a * x < x < b + x$ and so $a < b$.
\end{proof}

\begin{lemma}\label{split_first_and_last}\lean{split_first_and_last_factor_of_product}\leanok
Let $a$ and $b$ be elements of a semigroup $S$.

For all $n > 1$, $(a*b)^n = a * (b+a)^{n-1} * b$.
\end{lemma}
\begin{proof}
Let $n=2$. Then $(a+b)^2 = a * (b*a)^{2-1} * b$.

Assume that the statement holds for $n$.
Then we have that $(a*b)^{n+1} = a * b * (a*b)^n = a * b * a * (b*a)^{n-1} * b = a * (b*a)^n * b$.
\end{proof}

\begin{lemma}\label{thm:comm_ineq}\lean{comm_factor_le, comm_swap_le, comm_dist_le}\leanok
Let $a$ and $b$ be elements of an ordered semigroup $S$.

If $a * b \le b * a$, then for any $n\in \mathbb{N}^+$, we have that
\[a^n * b^n \le (a*b)^n \le (b*a)^n \le b^n * a^n\]
\end{lemma}
\begin{proof}
If $n=1$, then we are done.

Assume that the statement holds for $n$.
Then we have that
\begin{align}
a^{n+1} + b^{n+1} &= a * a^n * b^n * b \\
\text{by the induction hypothesis}\\
&< a * (a * b)^n * b \\
\text{since $a*b < b*a$}\\
&< a * (b * a)^n * b \\
\text{by the previous lemma}\\
&= (a * b)^{n+1} \\
&< (b * a)^{n+1} \\
\text{by the previous lemma}\\
&= b * (a * b)^n * a \\
&< b * (b * a)^n * a \\
\text{by the induction hypothesis}\\
&< b * b^n * a^n * a \\
&= b^{n+1} * a^{n+1}
\end{align}
\end{proof}

\begin{definition}\label{def:arch_wrt}\lean{is_archimedean_wrt}\leanok
\uses{def:positive, def:negative}
Let $a$ and $b$ be elements of an ordered semigroup $S$ that are not one.

Then $a$ is said to be \textbf{Archimedean with respect to} $b$
if there exists an $N\in \mathbb{N}^+$ such that for $n > N$,
the inequality $b < a^n$ holds if $b$ is positive,
and the inequality $b > a^n$ holds if $b$ is negative.
\end{definition}

\begin{definition}\label{def:arch}\lean{is_archimedean}\leanok
\uses{def:one, def:arch_wrt}
An ordered semigroup is \textbf{Archimedean} if any two of its elements
of the same sign are mutually Archimedean.
\end{definition}

\begin{definition}\label{def:anomalous_pair}\lean{anomalous_pair}\leanok
Let $a$ and $b$ be elements of an ordered semigroup $S$.

Then $a$ and $b$ are said to form an \textbf{anomalous pair}
if for all $n\in \mathbb{N}^+$, one of the following holds
\begin{align}
a^n < b^n < a^{n+1} \\
a^n > b^n > a^{n+1}
\end{align}
\end{definition}

\begin{theorem}\label{non_arch_anomalous}\lean{non_archimedean_anomalous_pair}\leanok
\uses{def:anomalous_pair, def:arch}
If $S$ is a non-Archimedean linear ordered cancel semigroup, then there exists an anomalous pair.
\end{theorem}
\begin{proof}
Since $S$ is non-Archimedean, there exists $a,b\in S$ such that
$a$ and $b$ have the sign and are not mutually Archimedean.

First, we look at the case where $a$ and $b$ are positive.
Then since they are not mutually Archimedean, without loss of generality, for all $n\in \mathbb{N}^+$,
$a^n < b$.

Then we either have that $a * b \le b * a$ or that $b*a\le a * b$.
In the first case, we have that
\[
a^n + b^n \le (a*b)^n \le b^n + a^n
\]
which means that, since $a^n < b$,
\[
b^n < (a*b)^n < b^{n+1}
\]
And so $b$ and $a*b$ form an anomalous pair.

The other cases follow similarly.
\end{proof}

\begin{theorem}\label{non_anomalous_comm_and_arch}\lean{not_anomalous_comm_and_arch}\leanok
\uses{def:anomalous_pair, def:arch}
A linear ordered cancel semigroup without anomalous pairs is an Archimedean commutative semigroup.
\end{theorem}
\begin{proof}
Let $a$ and $b$ be elements of an ordered semigroup $S$.
If either $a$ or $b$ is one, then they commute.

We begin with the case that $a$ and $b$ are positive.
If $a * b < b * a$, then for all $n\in \mathbb{N}^+$, we have that
\begin{align}
(a*b)^{n+1} &= a * (b*a)^n * b \\
&> (b*a)^n * b \\
&> (b*a)^n \\
&> (a*b)^n
\end{align}
And so $a*b$ and $b*a$ form an anomalous pair.

The same for the case that $b * a < a * b$.
Therefore, we must have that $a*b = b*a$.

The case where $a$ and $b$ are negative follows similarly.

We now look at the case where $a$ is negative and $b$ is positive.
If the element $1$ exists and $a*b = 1$, then $a * b * a = a$ and so $b*a = 1$.
Therefore, $a$ and $b$ commute.

If $a * b$ is positive, then
\begin{align}
(a*b) * (a*b) &> a * b \\
(b * a) * b &> b \\
b * a &> 0
\end{align}

We already showed that positive elements commute and so
\[ (b*a) * b = b * (b * a)\]

Now we look at the case where $a*b < b*a$.
Then we have that
\begin{align}
(a * b)^2 &= a * ((b*a) * b) \\
&= a * (b * (b * a)) \\
&= (a * b) * (b * a) \\
&> (a * b) * (a * b) \\
&= (a * b)^2
\end{align}
Which is a contradiction.
We get a similar contradiction in the case that $b*a < a*b$.
Therefore, $a*b = b*a$.

The case where $a*b$ is negative follows similarly.
The case where $b$ is negative and $a$ is positive is symmetric.
\end{proof}

\begin{definition}\label{def:has_large_differences}\lean{has_large_differences}\leanok
An ordered semigroup $S$ has \textbf{large differences} if
for all $a,b\in S$, the two following implications hold
\begin{itemize}
\item if $a$ is positive and $a<b$, then there exists $c\in S$
and $n\in \mathbb{N}^+$ such that $c$ is Archimedean with respect to $a$
and $a^n*c \le b^n$
\item if $a$ is negative and $b < a$, then there exists $c\in S$
and $n\in \mathbb{N}^+$ such that $c$ is Archimedean with respect to $a$
and $a^n*c \ge b^n$
\end{itemize}
\end{definition}

\begin{theorem}\lean{not_anomalous_iff_large_difference}\leanok
\uses{def:arch_wrt, def:anomalous_pair, def:has_large_differences}
In a linear ordered cancel semigroup $S$, there are no anomalous pairs
if and only if there are large differences.
\end{theorem}
\begin{proof}
($\Rightarrow$)
If $a$ and $b$ are positive and $a < b$, then we have that $a^n < b^n$ for all $n$.
Therefore, there must exist an $n$ such that $a^{n+1} \le b^n$
as otherwise $a$ and $b$ would form an anomalous pair.
Thus, the theorem is satisfied with $c = a$.

Similarly if $a$ and $b$ are negatiave.

($\Leftarrow$)
We look first at the case where $a$ and $b$ are positive and $a < b$.
Then we have $c\in S$ Archimedean with respect to $a$ and $m\in \mathbb{N}^+$ such that
$a^m * c \le b^m$.

Therefore, for any $n \in \mathbb{N}^+$,
either
\[(a^m)^n * c^n \le (a^m*c)^n \le (b^m)^n\]
or
\[c^n * (a^m)^n \le (a^m*c)^n \le (b^m)^n\]
holds.

Since $c$ is Archimedean with respect to $a$, there exists an $N$
such that for all $n \ge N$, $a < c^n$. Thus, for any $n \ge N$,
we get from the previous relations that
\[a^{mn + 1} \le b^{mn}\]
and so $a$ and $b$ do not form an anomalous pair.

Similalry if $a$ and $b$ are negative.
\end{proof}

\section{Holder's Theorem}
We follow the proof from ``Groups, Orders, and Dynamics'' by Deroin, Navas, and Rivas.

\begin{lemma}
Let $G$ be a left ordered group. 
If its positive cone is a normal subsemigroup, then $G$ is a bi-ordered group.
\end{lemma}
\begin{proof}
Let $x,y,z \in G$ such that $x \le y$. Since $G$ is a left ordered group, we then know that $zx \le zy$.
Our goal is to show that $xz \le yz$.

Since $x \le y$, we either have that $x = y$ or $x < y$.
In the first case, it is true that $xz = yz$ and so $xz \le yz$ and we are done.

In the second case, since $x < y$, we have that $1 < x^{-1}y$.
So $x^{-1}y$ is in the positive cone. Since the positive cone is normal, we have that
$1 < z^{-1}x^{-1}yz$. Therefore, $xz < yz$ and we are done.
\end{proof}

\begin{lemma}
Let $G$ be a left ordered group.
If $G$ is Archimedean, then it is bi-ordered.
\end{lemma}
\begin{proof}
By the previous lemma, it suffices to show that its positive cone is a normal semigroup.
Let $g$ be an element of the positive cone and let $h$ be an element of $G$.
Our goal is to show that $hgh^{-1}$ is in the positive cone.

Assume for the sake of contradiction that $hgh^{-1}$ is not in the positive cone.
We then split into two cases based on whether or not $h$ is in the positive or negative cone.
Notice that since $h \ne 1$, it must be in one of the cones.

Let's first look at the case where $h$ is in the negative cone.
Since $G$ is Archimedean, there exists a smallest nonnegative integer $n$
such that $h^{-1} < g^n$ (we can guarantee its nonnegative as both $g$ and $h^{-1}$ are positive).
Since $h < 1$, $1 < h^{-1}$ and so $n > 0$.

Notice that since $g$ and $h$ are not the identity,
and $hgh^{-1}$ is not in the positive cone, it must be in the negative cone.
So $hgh^{-1} < 1$ and thus, we have that $h^{-1} < g^{-1}h^{-1}$. And since
$h^{-1} < g^n$, we have that $g^{-1}h^{-1} < g^{n-1}$. Therefore,
we have that $h^{-1} < g^{-1}h^{-1} < g^{n-1}$. So then $h^{-1} < g^{n-1}$.
Since $n > 0$, $n-1$ is a smaller nonnegative integer such that $h^{-1} < g^{n-1}$. 
This contradicts the minimality of $n$.

The remaining case is where $h$ is in the positive cone.
As before, $hgh^{-1}$ must be in the negative cone so
$hgh^{-1} < 1$. Therefore, $1 < hg^{-1}h^{-1}$.

We have that $h^{-1}$ is in the negative cone
and so from the first half of this proof, we know that
$h^{-1}(hg^{-1}h^{-1})h$ is in the positive cone.
Simplifying, we get that $g^{-1}$ is in the positive cone.
So then we have that $1 < g^{-1}$ and $1 < g$ which is a contradiction.

So we have shown that the positive cone is a normal subsemigroup
and thus that the group $G$ is bi-ordered.
\end{proof}

\begin{theorem}
A left-ordered group that is Archimedean is isomorphic to a
subgroup of $\mathbb{R}$.
\end{theorem}
\begin{proof}
The first step is to define the following map:\\ \\
Let $G$ be a bi-ordered Archimedean group.
Fix a positive element $f \in G$. Then for each $g \in G$
we can define a function $q_g\colon \mathbb{N} \to \mathbb{N}$ where
for each $n \in \mathbb{N}$, $q_g(n)$ is the unique integer satisfying
\[
f^{q_g(n)} \le g^n < f^{q_g(n) + 1}
\]

The second step:\\ \\ 
We prove that for each $g \in G$,
the sequence $\frac{q_g(n)}{n}$ converges to a real number as $n$
goes to infinity.
Let $n_1$ and $n_2$ be natural numbers. Then
$f^{q_g(n_1)} \le g^{n_1} < f^{q_g(n_1) + 1}$ and
$f^{q_g(n_2)} \le g^{n_2} < f^{q_g(n_2) + 1}$.
Therefore,
\[
f^{q_g(n_1)+q_g(n_2)} \le g^{n_1+n_2} < f^{q_g(n_1) + q_g(n_2) + 2}
\]
And so it must be that 
\[
q_g(n_1) + q_g(n_2) \le q_g(n_1 + n_2) \le q_g(n_1) + q_g(n_2) + 1
\]
Then the sequence converges by the exercise already proven as we have that
$|q_g(n_1+n_2) - q_g(n_1) - q_g(n_2)| \le 1$.

The third step:\\ \\
We define a map $\phi$ that takes each element $g \in G$ to the real number which is the limit
of the sequence $\frac{q_g(n)}{n}$.

We then check that this map is a group homomorphism.
Let $g_1,g_2 \in G$. Our goal is to show that $\phi(g_1) + \phi(g_2) = \phi(g_1g_2)$.
We know that for any $p \in \mathbb{N}$, there exists $q_1$ and $q_2$ such that
\[
f^{q_1} \le g_1^p < f^{q_1 + 1}
\]
and
\[
f^{q_2} \le g_2^p < f^{q_2 + 1}
\]

We now do case work based on whether $g_1g_2 \le g_2g_1$ or
$g_2g_1 \le g_1g_2$. Let's look at the first case.
Then we have that $g_1^pg_2p \le (g_1g_2)^p \le g_2^pg_1^p$.
Furthermore we have that $f^{q_1+q_2} \le g_1^pg_2^p$ and that
$g_2^pg_1^p < f^{q_1+q_2+2}$. Therefore,
\[
f^{q_1+q_2} \le g_1^pg_2^p \le (g_1g_2)^p \le g_2^pg_1^p < f^{q_1+q_2+2}
\]
Therefore, for each $p$
\[
q_1+q_2 \le q_{g_1g_2}(p) \le q_1+q_2 + 1
\]
And so
\[
\lim_{p \to\infty} \frac{q_1+q_2}{p} \le \lim_{p\to\infty} \frac{q_{g_1g_2}(p)}{p}=\phi(g_1g_2) \le \lim_{p\to\infty} \frac{q_1+q_2 + 1}{p}
\]

Now we have that
\begin{align*}
\phi(g_1) + \phi(g_2) &= \lim_{p\to \infty} \frac{q_1}{p} + \lim_{p\to\infty} \frac{q_2}{p}\\
&= \lim_{p\to\infty} \frac{q_1+q_2}{p}\\
&\le \phi(g_1g_2)\\
&\le \lim_{p\to\infty} \frac{q_1+q_2 + 1}{p}\\
&= \lim_{p\to\infty} \frac{q_1+q_2}{p}\\
&= \phi(g_1)+\phi(g_2)
\end{align*}
Thus, $\phi(g_1g_2) = \phi(g_1) + \phi(g_2)$.
And the other case is the same.

The fourth step:\\ \\
We check that the map is injective and order-preserving.
\end{proof}