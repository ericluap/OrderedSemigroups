% In this file you should put the actual content of the blueprint.
% It will be used both by the web and the print version.
% It should *not* include the \begin{document}
%
% If you want to split the blueprint content into several files then
% the current file can be a simple sequence of \input. Otherwise It
% can start with a \section or \chapter for instance.

\chapter{Groups}
In this chapter, we prove Holder's theorem for groups. We follow the proof in
``Groups, Orders, and Dynamics'' by Deroin, Navas, and Rivas.

We choose an element of the ordered group $f$ and map it to $1$ in the real numbers.
Then for any other element $g$ of the ordered group, we use $f$ to construct
a sequence of rational approximations of $g$. We map $g$ to the real number that
is the limit of this sequence of approximations. We then prove that this map is
injective and order preserving.

\section{Definitions}
We begin with basic definitions of ordered groups.

\begin{definition}\label{def:LeftOrderedGroup}
    \lean{LeftOrderedGroup}\leanok
    A \textbf{left ordered group} $G$ is a group and a partial order such that for all $x, y, z \in G$,
    if $x \le y$, then $z * x \le z * y$.
\end{definition}

\begin{definition}\label{def:LeftLinearOrderedGroup}
    \lean{LeftLinearOrderedGroup}\uses{def:LeftOrderedGroup}\leanok
    A \textbf{left linear ordered group} $G$ is a left ordered group that is also a linear order.
\end{definition}

\begin{definition}\label{def:archimedean_group}
    \lean{archimedean_group}\uses{def:LeftOrderedGroup}\leanok
    An \textbf{Archimedean group} is a left ordered group such that
    for any $g, h \in G$ where $g \ne 1$, there exists an integer $z$
    such that $h < g^z$.
\end{definition}

\section{Approximation}
In this section we assume that $G$ is a left linear ordered group
that is Archimedean. Furthermore, we assume we have an element $f\in G$
such that $1 < f$.
\begin{theorem}\label{approximate}
    \lean{approximate}\leanok
    \uses{def:LefLinearOrderedGroup, def:archimedean_group}
    For any $g \in G$ and $p \in \mathbb{N}$,
    there exists an integer $q \in \mathbb{Z}$
    such that
    \[
        f^q \le g^p < f^{q+1}
    \]
\end{theorem}
\begin{proof}\leanok
Since $G$ is an Archimedean group, we can construct exponents $l$ and $u$
such that $f^l < g^p < f^u$. Therefore, there must exist some integer $q$
which satisfies what we want.
\end{proof}

\begin{definition}\label{def:q}
    \lean{q}\leanok\uses{approximate}
    We define a function $q \colon G \to \mathbb{N} \to \mathbb{R}$
    using \Cref{approximate} such that for any $g\in G$ and $n \in \N$,
    \[
        f^{q_g(n)} \le g^n < f^{q_g(n)}
    \]
\end{definition}

\begin{theorem}\label{sequence_convergence}
    \lean{sequence_convergence}\leanok
    For any sequence $a_n$ of real numbers,
    if there exists $C \in \R$ such that for all $m,n\in \N$
    we have that
    \[
        |a_{m+n} - a_m - a_n| \le C
    \]
    then sequence $\frac{a_n}{n}$ converges.
\end{theorem}

\begin{theorem}\label{q_exp_add}
    \lean{q_exp_add}\leanok\uses{def:q}
    For any $g \in G$ and $a,b \in \mathbb{N}$, we have that
    \[
        f^{q_g(a) + q_g(b)} \le g^{a+b} < f^{q_g(a) + q_g(b) + 2}
    \]
\end{theorem}
\begin{proof}\leanok
We know the following two things by the definition of $q$
\begin{align*}
f^{q_g(a)}\le &g^a < f^{q_g(a)+1}\\
f^{q_g(b)}\le &g^b < f^{q_g(b)+1}\\
\end{align*}
And so it follows that
\[
f^{q_g(a)+q_g(b)}\le g^{a+b} < f^{q_g(a)+q_g(b) + 2}
\]
\end{proof}

\begin{theorem}\label{q_convergence}
    \lean{q_convergence}\leanok\uses{sequence_convergence, def:q, q_exp_add}
    For any $g \in G$, the sequence $\frac{q_g(n)}{n}$ converges.
\end{theorem}
\begin{proof}\leanok
From \Cref{q_exp_add}, we have that
\[
q_g(a) + q_g(b) \le q_g(a+b) \le q_g(a)+q_g(b) + 1
\]
and so
\[
|q_g(a+b) - q_g(a) - q_g(b)| \le 1
\]
Therefore, by \Cref{sequence_convergence} with $C = 1$, we have that
the sequence $\frac{q_g(n)}{n}$ converges.
\end{proof}

\section{Map}
We make the same assumptions as in the previous section.
So we assume that $G$ is a left linear ordered group
that is Archimedean, $f\in G$, and $1 < f$.

We now define the map from the $G$ to $\R$ and prove its properties.

\begin{definition}\label{def:phi}
    \lean{φ'}\leanok\uses{q_convergence}
    We define a map $\phi\colon G \to \R$ by mapping $g$
    to the real number that $\frac{q_g(n)}{n}$ converges to
    as we know from \Cref{q_convergence}.
\end{definition}

\begin{theorem}\label{phi_hom_case}
    \lean{φ'_hom_case}\leanok\uses{def:q}
    For all $g_1,g_2 \in G$ and $p\in \mathbb{N}$,
    \[
    q_{g_1}(p) + q_{g_2}(p) \le q_{g_1g_2}(p) \le q_{g_1}(p) + q_{g_2}(p) + 1
    \]
\end{theorem}
\begin{proof}\leanok
Let $q_1 = q_{g_1}(p)$ and $q_2 = q_{g_2}(p)$. Then we know that
\begin{align*}
f^{q_1} \le &g_1^p < f^{q_1+1}\\
f^{q_2} \le &g_2^p < f^{q_2+1}\\
\end{align*}
And so we also have the following two facts
\begin{align*}
f^{q_1+q_2}&\le g_1^p g_2^p\\
g_2^p g_1^p& < f^{q_1 + q_2 + 2}
\end{align*}

We look at the case where $g_1g_2 \le g_2g_1$.
Then $g_1^p g_2^p \le (g_1g_2)^p \le g_2^p g_1^p$.
And so combined with the previous facts, we have that
\[
f^{q_1+q_2}\le g_1^p g_2^p \le (g_1g_2)^p \le g_2^p g_1^p < f^{q_1+q_2+2}
\]

Therefore,
\[
q_1+q_2 \le q_{g_1g_2}(p) \le q_1+q_2+1
\]
And so we are done.
The case where $g_2g_1 \le g_1g_2$ follows similarly.
\end{proof}

\begin{theorem}\label{phi_hom}
    \lean{φ'_hom}\leanok\uses{def:phi, phi_hom_case}
    The map $\phi$ is a homomorphism.
\end{theorem}
\begin{proof}\leanok
Let $g_1, g_2 \in G$. Then from \Cref{phi_hom_case} we have that
\[
    q_{g_1}(p) + q_{g_2}(p) \le q_{g_1g_2}(p) \le q_{g_1}(p) + q_{g_2}(p) + 1
\]
And so since $\lim_{p\to\infty}\frac{q_{g_1}(p) + q_{g_2}(p)}{p} = \lim_{p\to\infty}\frac{q_{g_1}(p) + q_{g_2}(p) + 1}{p}$,
we have that
\[
\lim_{p\to\infty}\frac{q_{g_1}(p) + q_{g_2}(p)}{p} = \lim_{p\to\infty}\frac{q_{g_1g_2(p)}}{p}
\]
Therefore, by the definition of $\phi$, we have shown that $\phi(g_1)+\phi(g_2) = \phi(g_1g_2)$.
\end{proof}

\begin{theorem}\label{order_preserving_phi}
    \lean{order_preserving_φ}\leanok\uses{def:phi}
    For all $g,h \in G$, if $g \le h$ then $\phi(g) \le \phi(h)$.
\end{theorem}
\begin{proof}\leanok
First, notice that since $g\le h$, then for all $p \in \mathbb{N}$,
$q_g(p) \le q_h(p)$. Then from the definition of $\phi$, it follows that
$\phi(g) \le \phi(h)$.
\end{proof}

\begin{theorem}\label{f_maps_one}
    \lean{f_maps_one_φ}\leanok\uses{def:phi}
    We have that $\phi(f) = 1$ where $f$ is our fixed
    positive element.
\end{theorem}
\begin{proof}\leanok
We have that for all $n\in \mathbb{n}$ that $f^n \le f^n < f^{n+1}$
and so $q_f(n) = n$. Therefore, $\phi(f) = 1$.
\end{proof}

\begin{theorem}\label{injective_phi}
    \lean{injective_φ}\leanok
    \uses{f_maps_one, order_preserving_phi, phi_hom}
    The map $\phi$ is injective.
\end{theorem}
\begin{proof}\leanok
Since from \Cref{phi_hom} we have that $\phi$ is a homomorphism,
it suffices to show that for any $g\in G$, if $\phi(g) = 0$, then $g = 1$.

Assume for the sake of contradiction that there exists $g \in G$
such that $\phi(g) = 0$ but $g$ is not equal to $1$.
Then since $G$ is Archimedean, there exists an integer $z$
such that $g^z > f$. Therefore, since by \Cref{f_maps_one}
we have that $\phi(f) = 1$,
\begin{align*}
0 &= \phi(g) = \phi(g)^z = \phi(g^z)\\
&> \phi(f) = 1
\end{align*}
Contradiction.
\end{proof}

\begin{theorem}\label{strict_order_preserving_phi}
    \lean{strict_order_preserving_φ}\leanok
    \uses{order_preserving_phi, injective_phi}
    For all $g,h \in G$, we have that
    $g \le h$ if and only if $\phi(g) \le \phi(h)$.
\end{theorem}
\begin{proof}\leanok
($\Rightarrow$) This is \Cref{order_preserving_phi}.

($\Leftarrow$) We have that $\phi(g) \le \phi(h)$.
Assume for the sake of contradiction that $h < g$.
Then by \Cref{order_preserving_phi}, we know that
$\phi(h) \le \phi(g)$. Therefore, $\phi(g) = \phi(h)$.
And so by \Cref{injective_phi}, we know that $\phi$ is injective and
so $g = h$. Contradiction.
\end{proof}

\section{Holder's Theorem}
\begin{theorem}\label{holders_theorem}
    \lean{holders_theorem}\leanok
    \uses{def:phi, injective_phi, strict_order_preserving_phi}
    If $G$ is a left linear ordered group that is Archimedean,
    then $G$ is isomorphic to a subgroup of $\mathbb{R}$.
\end{theorem}
\begin{proof}\leanok
First, we look at the case where there exists a positive element $f$ in $G$.
Given such an element, we have an order preserving, injective homomorphism
$\phi$. And so $G$ is isomorphic to the image of $\phi$ which is a subgroup of $\mathbb{R}$.

If there does not exist a positive element in $G$,
then $G$ is trivial and is isomorphic to the trivial subgroup of
$\mathbb{R}$.
\end{proof}

\chapter{Semigroups}
We follow the paper ``On ordered semigroups'' by
N. G. Alimov.

We show that if a a linear ordered cancel semigroup
does not have anomalous pairs, then there exists a larger
Archimedean group containing it. Then from Holder's theorem for groups,
that larger group is isomorphic to a subgroup of $\mathbb{R}$
and so the smaller semigroup is isomorphic to a subsemigroup of $\mathbb{R}$.

\section{Definitions}

\begin{definition}\label{def:LeftOrderedSemigroup}
    \lean{LeftOrderedSemigroup}\leanok
    A \textbf{left ordered semigroup} $S$ is a semigroup and a partial order
    such that for all $x, y, z\in S$,
    if $x \le y$, then $z * x \le z * y$.
\end{definition}

\begin{definition}\label{def:RightOrderedSemigroup}
    \lean{RightOrderedSemigroup}\leanok
    A \textbf{right ordered semigroup} $S$ is a semigroup and a partial order
    such that for all $x, y, z\in S$,
    if $x \le y$, then $x * z \le y * z$.
\end{definition}

\begin{definition}\label{def:OrderedSemigroup}
    \lean{OrderedSemigroup}\leanok
    \uses{def:LeftOrderedSemigroup, def:RightOrderedSemigroup}
An \textbf{ordered semigroup} $S$ is a
left and right ordered semigroup.
\end{definition}

\begin{definition}\label{def:OrderedCancelSemigroup}\lean{OrderedCancelSemigroup}\leanok
\uses{def:OrderedSemigroup}
An \textbf{ordered cancel semigroup} $S$ is an ordered semigroup such that for all $a,b,c\in S$,
if $a * b \le a * c$ then $b \le c$ and if $b * a \le c * a$ then $b \le c$.
\end{definition}

\begin{definition}\label{def:LinearOrderedSemigroup}\lean{LinearOrderedSemigroup}\leanok
\uses{def:OrderedSemigroup}
A \textbf{linear ordered semigroup} is an ordered semigroup where its partial order is a linear order.
\end{definition}

\begin{definition}\label{def:LinearOrderedCancelSemigroup}\lean{LinearOrderedCancelSemigroup}\leanok
\uses{def:OrderedCancelSemigroup}
A \textbf{linear ordered cancel semigroup} is an ordered cancel semigroup where its partial order is a linear order.
\end{definition}

\begin{definition}\label{def:anomalous_pair}
    \lean{anomalous_pair}\leanok
    An \textbf{anomalous pair} in a left ordered semigroup $S$
    is a pair of elements $a,b \in S$ such that
    for all positive $n \in \mathbb{N}$, either
    \[
        a^n < b^n < a^{n+1}
    \]
    or
    \[
        a^n > b^n > a^{n+1}
    \].
\end{definition}

Intuitively, an anomalous pair represents a pair
of elements that are infinitely close.
If you have two real numbers $s$ and $r$ such that $s < r$,
then as $n \in \N$ gets larger, $n\cdot s$ and $n\cdot r$
get farther apart. However, for an anomalous pair,
the elements alawys remain close to each other.

\begin{definition}\label{def:positive}\lean{is_positive}\leanok
\uses{def:LeftOrderedSemigroup}
An element $a$ of a left ordered semigroup $S$ is \textbf{positive} if for all $x\in S$, $a*x > x$.
\end{definition}

\begin{definition}\label{def:negative}\lean{is_negative}\leanok
\uses{def:LeftOrderedSemigroup}
An element $a$ of a left ordered semigroup $S$
is \textbf{negative} if for all $x\in S$, $a*x < x$.
\end{definition}

\begin{definition}\label{def:one}\lean{is_one}\leanok
\uses{def:LeftOrderedSemigroup}
An element $a$ of a left ordered semigroup $S$
is \textbf{one} if for all $x\in S$, $a*x = x$.
\end{definition}

\begin{definition}\label{def:arch_wrt}\lean{is_archimedean_wrt}\leanok
\uses{def:positive, def:negative}
Let $a$ and $b$ be elements of a left ordered semigroup $S$ that are not one.

Then $a$ is said to be \textbf{Archimedean with respect to} $b$
if there exists an $N\in \mathbb{N}^+$ such that for $n > N$,
the inequality $b < a^n$ holds if $b$ is positive,
and the inequality $b > a^n$ holds if $b$ is negative.
\end{definition}

\begin{definition}\label{def:arch}\lean{is_archimedean}\leanok
\uses{def:one, def:arch_wrt}
A left ordered semigroup is \textbf{Archimedean} if any two of its elements
of the same sign are mutually Archimedean.
\end{definition}

\begin{definition}\label{def:has_large_differences}\lean{has_large_differences}\leanok
    \uses{def:arch_wrt, def:positive, def:negative}
A left ordered semigroup $S$ has \textbf{large differences} if
for all $a,b\in S$, the two following implications hold
\begin{itemize}
\item if $a$ is positive and $a<b$, then there exists $c\in S$
and $n\in \mathbb{N}^+$ such that $c$ is Archimedean with respect to $a$
and $a^n*c \le b^n$
\item if $a$ is negative and $b < a$, then there exists $c\in S$
and $n\in \mathbb{N}^+$ such that $c$ is Archimedean with respect to $a$
and $a^n*c \ge b^n$
\end{itemize}
\end{definition}

Intuitively, this is saying if $a < b$ then eventually $a^n$
is significantly smaller than $b^n$. Here ``significantly smaller''
means that there is an element that is not infinitely small
with respect to $a$ that separates $a^n$ and $b$.

\section{Anomalous Pairs}

\begin{theorem}\label{thm:pos_neg_or_one}\lean{pos_neg_or_one}\leanok
    \uses{def:positive, def:negative, def:one, def:LinearOrderedCancelSemigroup}
Each element $a$ of a linear ordered cancel semigroup $S$ is either positive, negative, or one.
\end{theorem}
\begin{proof}\leanok
Let $a\in S$ and $b\in S$. Since the $S$ is a linear order we have one of the following cases.

The first case is that $b * a > b$. Then for all $x\in S$ we have that $b * a * x > b * x$
and so $a * x > x$. Therefore, $a$ is positive.

The second case is that $b * a < b$. Then for all $x\in S$ we have that $b * a * x < b * x$
and so $a * x < x$. Therefore, $a$ is negative.

The last case is that $b * a = b$. Then for all $x\in S$ we have that $b * a * x = b * x$
and so $a * x = x$. Therefore, $a$ is zero.
\end{proof}

\begin{theorem}\label{non_arch_anomalous}\lean{non_archimedean_anomalous_pair}\leanok
\uses{def:anomalous_pair, def:arch}
If $S$ is a non-Archimedean linear ordered cancel semigroup, then there exists an anomalous pair.
\end{theorem}
\begin{proof}\leanok
Since $S$ is non-Archimedean, there exists $a,b\in S$ such that
$a$ and $b$ have the sign and are not mutually Archimedean.

First, we look at the case where $a$ and $b$ are positive.
Then since they are not mutually Archimedean, without loss of generality, for all $n\in \mathbb{N}^+$,
$a^n < b$.

Then we either have that $a * b \le b * a$ or that $b*a\le a * b$.
In the first case, we have that
\[
a^n + b^n \le (a*b)^n \le b^n + a^n
\]
which means that, since $a^n < b$,
\[
b^n < (a*b)^n < b^{n+1}
\]
And so $b$ and $a*b$ form an anomalous pair.

The other cases follow similarly.
\end{proof}

\begin{theorem}\label{non_anomalous_comm_and_arch}\lean{not_anomalous_comm_and_arch}\leanok
\uses{def:anomalous_pair, def:arch}
A linear ordered cancel semigroup without anomalous pairs is an Archimedean commutative semigroup.
\end{theorem}
\begin{proof}\leanok
Let $a$ and $b$ be elements of an ordered semigroup $S$.
If either $a$ or $b$ is one, then they commute.

We begin with the case that $a$ and $b$ are positive.
If $a * b < b * a$, then for all $n\in \mathbb{N}^+$, we have that
\begin{align*}
(a*b)^{n+1} &= a * (b*a)^n * b \\
&> (b*a)^n * b \\
&> (b*a)^n \\
&> (a*b)^n
\end{align*}
And so $a*b$ and $b*a$ form an anomalous pair.

The same for the case that $b * a < a * b$.
Therefore, we must have that $a*b = b*a$.

The case where $a$ and $b$ are negative follows similarly.

We now look at the case where $a$ is negative and $b$ is positive.
If the element $1$ exists and $a*b = 1$, then $a * b * a = a$ and so $b*a = 1$.
Therefore, $a$ and $b$ commute.

If $a * b$ is positive, then
\begin{align*}
(a*b) * (a*b) &> a * b \\
(b * a) * b &> b \\
b * a &> 0
\end{align*}

We already showed that positive elements commute and so
\[ (b*a) * b = b * (b * a)\]

Now we look at the case where $a*b < b*a$.
Then we have that
\begin{align*}
(a * b)^2 &= a * ((b*a) * b) \\
&= a * (b * (b * a)) \\
&= (a * b) * (b * a) \\
&> (a * b) * (a * b) \\
&= (a * b)^2
\end{align*}
Which is a contradiction.
We get a similar contradiction in the case that $b*a < a*b$.
Therefore, $a*b = b*a$.

The case where $a*b$ is negative follows similarly.
The case where $b$ is negative and $a$ is positive is symmetric.
\end{proof}

\begin{theorem}\label{not_anomalous_iff_large_difference}
    \lean{not_anomalous_iff_large_difference}\leanok
\uses{def:arch_wrt, def:anomalous_pair, def:has_large_differences}
In a linear ordered cancel semigroup $S$, there are no anomalous pairs
if and only if there are large differences.
\end{theorem}
\begin{proof}\leanok
($\Rightarrow$)
If $a$ and $b$ are positive and $a < b$, then we have that $a^n < b^n$ for all $n$.
Therefore, there must exist an $n$ such that $a^{n+1} \le b^n$
as otherwise $a$ and $b$ would form an anomalous pair.
Thus, the theorem is satisfied with $c = a$.

Similarly if $a$ and $b$ are negatiave.

($\Leftarrow$)
We look first at the case where $a$ and $b$ are positive and $a < b$.
Then we have $c\in S$ Archimedean with respect to $a$ and $m\in \mathbb{N}^+$ such that
$a^m * c \le b^m$.

Therefore, for any $n \in \mathbb{N}^+$,
either
\[(a^m)^n * c^n \le (a^m*c)^n \le (b^m)^n\]
or
\[c^n * (a^m)^n \le (a^m*c)^n \le (b^m)^n\]
holds.

Since $c$ is Archimedean with respect to $a$, there exists an $N$
such that for all $n \ge N$, $a < c^n$. Thus, for any $n \ge N$,
we get from the previous relations that
\[a^{mn + 1} \le b^{mn}\]
and so $a$ and $b$ do not form an anomalous pair.

Similalry if $a$ and $b$ are negative.
\end{proof}

\section{Semigroup to Group}

\begin{theorem}\label{to_not_anom_monoid}
    \lean{to_not_anom_monoid}\leanok
    \uses{def:anomalous_pair, def:LinearOrderedCancelSemigroup, non_anomalous_comm_and_arch}
    If $S$ is a linear ordered cancel semigroup
    without anomalous pairs, then there exists a
    linear ordered cancel commutative monoid $M$
    without anomalous pairs such that $S$ is isomorphic
    to a subsemigroup of $M$.
\end{theorem}
\begin{proof}\leanok
We do casework on whether or not $S$ has an element
that is one.

If $S$ has an element that is one then it is already
a monoid. Furthermore, since it has no anomalous pairs,
by \Cref{non_anomalous_comm_and_arch}, it is commutative.
And so we are done.

If $S$ does not have an element that is one,
then we let $M$ be $S$ with one added.
We define one to be less than every positive element and
greater than every negative element. By \Cref{non_anomalous_comm_and_arch},
we know that $S$ is commutative and so $M$ is too.
Furthermore, it is clear that $M$ has no anomalous pairs still.
Then $S$ is isomorphic to the subsemigroup of $M$
that is all its elements except for the added one.
\end{proof}

\begin{theorem}\label{pos_large_elements}
    \lean{pos_large_elements, neg_large_elements}\leanok
    \uses{non_anomalous_comm_and_arch}
    Let $S$ be a linear ordered cancel semigroup without anomalous pairs
    such that there exists a positive element of $S$.
    Then for all $x \in S$, there exists a $y \in S$
    such that $y$ is positive and $x*y$ is positive.

    Similarly, if there exists a negative element of $S$
    then for all $x \in S$ there exists a $y \in S$
    such that $y$ is negative and $x*y$ is negative.
\end{theorem}
\begin{proof}\leanok
We look at the positive case. If $x$ is positive then
we are done as $x*x$ is positive.

Next we look at the case where $x$ is negative.
Assume for the sake of contradiction that for all $y \in S$,
if $y$ is positive then $x*y$ is negative. Let $z$
be a positive element we assumed existed in $S$.
Then for all positive $n \in \N$, we have that
$x * z^{n+2}$ is negative. 
Recall that from \Cref{non_anomalous_comm_and_arch} we have
commutativity.
From there we have an anomalous pair:
\begin{align*}
(x * z)^n &= x^n * z^n \ge x^n
&\ge x^n * (x * z^{n+2})\\
&= (x * z)^{n+1} * z > (x*z)^{n+1}
\end{align*}
Contradiction. And similarly for the negative case.
\end{proof}

\begin{theorem}\label{not_anom_pos_pair}
    \lean{not_anom_pos_pair}\leanok
    \uses{pos_large_elements, neg_large_elements}
    Let $M$ be a linear ordered cancel commutative monoid
    without anomalous pairs and let $G$ be its Grothendieck group.
    If $M$ has a positive (negative) element, then
    for any element $\frac{a}{b} \in G$ where $a,b \in M$,
    there exist $a', b' \in M$ such that $a'$ and $b'$ are positive (negative)
    and $\frac{a}{b} = \frac{a'}{b'}$.
\end{theorem}
\begin{proof}\leanok
Let $\frac{a}{b}\in G$ where $a,b\in M$.
Since $M$ has a positive element, by \Cref{pos_large_elements},
we have $a_2, b_2\in M$ such that $a*a_2$ and $b*b_2$ are positive.
Let $a' = a*a_2$ and $b' = b*b_2$. Then $\frac{a}{b} = \frac{a'}{b'}$
and $a'$ and $b'$ are positive.

And similarly for the negative case.
\end{proof}

\begin{theorem}\label{not_anom_to_arch}
    \lean{not_anom_to_arch}\leanok
    \uses{not_anom_pos_pair, non_anomalous_comm_and_arch}
    Let $M$ be a linear ordered cancel commutative
    monoid. If $M$ does not have anomalous pairs,
    then its Grothendieck group is Archimedean.
\end{theorem}
\begin{proof}\leanok
If $M$ is trivial then we are done.
Otherwise, it has a positive element of a negative element.

We look at the case where $M$ has a positive element $t$.
We now want to show that the Grothendieck group $G$ of $M$
is Archimedean. It suffices to show that for any positive
$g, h \in G$, there exists an integer $n$ such that $g^n > h$.
Since $g, h \in G$, there exist $g_1,g_2,h_1,h_2 \in M$ such that
$g = \frac{g_1}{g_2}$ and $h = \frac{h_1}{h_2}$, and
by \Cref{not_anom_pos_pair}, we can assume that
$g_1$, $g_2$, $h_1$, and $h_2$ are positive.

Then since $M$ does not have anomalous pairs, we have by
\Cref{non_anomalous_comm_and_arch} that $M$ is Archimedean.
Since $g_1$ and $h_1$ are positive and $M$ is Archimedean,
there exists a positive $N \in \mathbb{N}$ such that
for all $n \ge N$, $g_2^n > h_1$. So in particular,
we have that $g_2^N > h_1$.

Notice that since $\frac{g_1}{g_2}$ positive, we have that $g_2 < g_1$.
And therefore, since $M$ does not have anomalous pairs, there
exists a positive natural number $N_1$ such that
$g_2^{N_1+N} < g_1^{N_1}$.

We now claim that $g^{N_1} > h$. This is equivalent to
showing that $g_2^{N_1}*h_1 < h_2*g_1^{N_1}$. 
And we have that
\begin{align*}
g_2^{N_1} * h_1 &= h_1 * g_2^{N_1}\\
&< g_2^N * g_2^{N_1} = g_2^{N_1 + N}\\
&< g_1^{N_1} \\
&< h_1 * g_1^{N_1}
\end{align*}
And so we are done.

The final case where $M$ has a negative element
follows similarly.
\end{proof}

\begin{theorem}\label{to_arch_group}
    \lean{to_arch_group}\leanok
    \uses{to_not_anom_monoid, def:anomalous_pair, not_anom_to_arch}
    If $S$ is a linear ordered cancel semigroup
    that does not have anomalous pairs, then
    there exists a linear ordered commutative group $G$
    that is Archimedean
    such that $S$ is isomorphic to a subsemigroup of $G$.
\end{theorem}
\begin{proof}\leanok
By \Cref{to_not_anom_monoid}, we have that
$S$ is isomorphic to a subsemigroup of a
linear ordered cancel commutative monoid $M$ that
does not have anomalous pairs.

We let $G$ be the Grothendieck group of $M$.
Then by \Cref{not_anom_to_arch} we know that $G$
is Archimedean. Since $G$ is the Grothendieck group of $M$,
$M$ is isomorphic to a submonoid of $G$. Thus,
we have that $S$ is isomorphic to a subsemigroup of $G$.
\end{proof}

\section{Holder's Theorem}

\begin{theorem}\label{holder_not_anom}
    \lean{holder_not_anom}\leanok
    \uses{to_arch_group, holders_theorem}
    Let $S$ be a linear ordered cancel semigroup
    with out anomalous pairs. Then $S$ is isomorphic
    to a subsemigroup of the real numbers.
\end{theorem}
\begin{proof}\leanok
By \Cref{to_arch_group}, there exists a linear ordered commutative
Archimedean group $G$
such that $S$ is isomorphic to a subsemigroup of $G$.
By \Cref{holders_theorem}, $G$ is isomorphic to a subgroup of $\R$.
Therefore, $S$ is isomorphic to a subsemigroup of $\R$.
\end{proof}